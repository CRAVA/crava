\chapter{User guide}
\label{sec:userguide}
\index{CRAVA!userguide}

In this chapter, we describe how to build a \crava model file. The model file mainly follows the XML-format, but we also use the character '\#' for commenting, meaning that the rest of the line after such a character is read as comment. All model files start with \kw{crava}, and end with \texttt{<\\crava>}.

\section{Basic inversion}
\label{sec:basicinv}
\index{inversion!basic}
A primary ability for \crava is to run simple first-pass inversions. In this section, we describe how to build a model file for a simple inversion.  
\subsection{Survey information}
\subsubsection{Seismic data}
\subsubsection{Wavelet}
\subsubsection{Noise}
\subsection{Inversion volume}
\subsubsection{Top and bottom surfaces}
\subsubsection{Lateral extent}
\subsubsection{Depth conversion}
\subsection{Background model and correlations}
\subsubsection{Background model}
\subsubsection{Correlations}
\subsection{Well data}
\subsection{Output}
\subsubsection{Grid output}
\subsubsection{Well output}
\subsection{Tasks}
%inversion, prediction/simulation
\section{Advanced inversion options}
\subsection{Non-stationary wavelet and noise}
\subsection{PS-seismic and reflection approximations}
\subsection{Well quality checks}
\subsection{Controlling lateral correlation}
\subsection{Depth conversion}
\subsection{Miscellaneous}
\section{Estimation}
\subsection{Wavelet estimation}
\subsection{Noise estimation}
\subsection{Background model estimation}
\section{Facies prediction}
\subsection{Prior probabilities}
\subsection{Relative versus absolute elastic parameters}
\section{Forward modelling}
