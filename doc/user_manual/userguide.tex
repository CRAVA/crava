\chapter{User guide}
\label{sec:userguide}
\index{CRAVA!userguide}

In this chapter, we describe how to build a \crava model file. The model file mainly follows the XML-format, but we also use the character '\#' for commenting, meaning that the rest of the line after such a character is read as comment. XML/files are built with start- and endtags, encapsulating toher tags or values. All model files start with \texttt{<crava>}, and end with \texttt{<\\crava>}. An example of a model file is given in NBNB.

\section{Basic inversion}
\label{sec:basicinv}
\index{inversion!basic}
A primary ability for \crava is to run simple first-pass inversions. In this section, we describe how to build a model file for a simple inversion. We focus on how to get the key information into the program, whereas more detailed controls are discussed later, in \autoref{sec:advinv}. The key information elements for a \crava inversion run is:
\begin{itemize}
\item \hyperref[sec:basicseis]{Seismic data}.
\item \hyperref[sec:basicwave]{Wavelet}.
\item \hyperref[sec:basicnoise]{Signal/noise ratio}.
\item \hyperref[sec:basicvol]{Inversion volume}.
\item \hyperref[sec:basicbg]{Background model}.
\item \hyperref[sec:basiccorr]{Correlation structures}.
\end{itemize}.
Since \crava is designed to estimate any information that is not given, well data must also commonly be included.

\subsection{Survey information}
All information regarding the seismic data is gathered under the \kw{survey} tag. This includes location of seismic data files, wavelet information and signal-to-noise ratio for each angle gather. As an example, it may look like this:
\begin{example}
<survey>
  <segy-start-time>  2500.0 </segy-start-time>

  <angle-gather>
    <offset-angle>     16.0 </offset-angle>
    <seismic-data>
      <file-name> ../input/seismic/Cube16.segy </file-name>
    </seismic-data>
  </angle-gather>

  <angle-gather>
    <offset-angle>     28.0 </offset-angle>
    <seismic-data>
      <file-name> ../input/seismic/Cube28.segy </file-name>
    </seismic-data>
  </angle-gather>
</survey>
\end{example}

The seismic data should be on SegY-format, with a common offset time, given with \kw{segy-start-time} if different from 0. The first value is used to represent the interval from start-time to start-time + time-step, so with a start-time of 100ms, and 4ms sampling, the first value is used in the grid cell covering the interval 100-104ms. 

For each available angle, the rest of the information is gathered under an \kw{angle-gather} tag, one for each offset. The actual angle is given by \kw{offset-angle}.
\subsubsection{Seismic data}
\subsubsection{Wavelet}
\subsubsection{Signal/nois ratio}
\subsection{Inversion volume}
\subsubsection{Top and bottom surfaces}
\subsubsection{Lateral extent}
\subsubsection{Depth conversion}
\subsection{Background model and correlations}
\subsubsection{Background model}
\subsubsection{Correlations}
\subsection{Well data}
\subsection{Output}
\subsubsection{Grid output}
\subsubsection{Well output}
\subsection{Tasks}
%inversion, prediction/simulation
\section{Advanced inversion options}
\subsection{Non-stationary wavelet and noise}
\subsection{PS-seismic and reflection approximations}
\subsection{Well quality checks}
\subsection{Controlling lateral correlation}
\subsection{Depth conversion}
\subsection{Miscellaneous}
\section{Estimation}
\subsection{Wavelet estimation}
\subsection{Noise estimation}
\subsection{Background model estimation}
\section{Facies prediction}
\subsection{Prior probabilities}
\subsection{Relative versus absolute elastic parameters}
\section{Forward modelling}
