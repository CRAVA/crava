
\chapter{CRAVA model file reference manual}\index{reference manual, CRAVA model file elements}
\label{ap:model-file-reference}
\index{CRAVA model file@Crava model file!reference manual for elements}
The numbering shows the command grouping. A command with no subnumbering expects a value to be given, otherwise, it is only a grouping of other commands.

File names are currently given with a path relative to the directory settings in <project-settings>-<io-settings>-<input/output/top-directory>. If these are not given, the path will always be relative to the working directory.

All commands are optional, unless otherwise stated. A necessary command under an optional is only necessary if the optional is given.


\section{\hbracket{actions}\necessary} \newkw{actions}
 \slist
   \item \Description Controls the main purpose of the run.
   \item \Argument
   \item \Default
 \elist

\subsection{\hbracket{mode}}  \newkw{mode}
 \slist
   \item \Description Inversion: Invert seismic input data to elastic parameters and/or facies probabilities. Needs seismic data and volume, all other missing data will be estimated.
Forward: Create seismic response from background model. Not able to estimate anything.
Estimation: Checks input data and performs estimation of lacking information for inversion, but sops before inversion.
   \item \Argument 'inversion', 'forward' or 'estimation'
   \item \Default
 \elist

\subsection{\hbracket{inversion-settings}}  \newkw{inversion-settings}
 \slist
   \item \Description Only valid with the 'inversion' choice above. Controls aspects of the inversion.
   \item \Argument Elements for different inversion settings.
   \item \Default
 \elist

\subsubsection{\hbracket{prediction}}  \newkw{prediction}
 \slist
   \item \Description Controls whether predicted elastic parameters will be generated.
   \item \Argument 'yes' or 'no'.
   \item \Default
 \elist

\subsubsection{\hbracket{simulation}}  \newkw{simulation}
 \slist
   \item \Description Controls aspects of the simulation of elastic parameters.
   \item \Argument
   \item \Default
 \elist

\paragraph{cs}

\paragraph{\hbracket{seed}}  \newkw{seed}
 \slist
   \item \Description A number used to initialize the random generator. Running a model file with a given seed will give the same simulation results each time.
   \item \Argument
   \item \Default
 \elist

\paragraph{\hbracket{seed-file}}  \newkw{seed-file}
 \slist
   \item \Description Alternative to \kw{seed}. This is an ASCII file containing a number. At the termination of the run, the file will be overwritten with a seed generated by the random generator. Thus, a model file using this will generate different simulation results on sequential runs.
   \item \Argument
   \item \Default
 \elist

\paragraph{\hbracket{number-of-simulations}}  \newkw{number-of-simulations}
 \slist
   \item \Description Integer value giving the number of stochastic realizations to generate.
   \item \Argument
   \item \Default
 \elist

\subsubsection{\hbracket{conditioning-to-wells}}  \newkw{conditioning-to-wells}
 \slist
   \item \Description Should the realizations be kriged to well data?
   \item \Argument 'yes' or 'no'
   \item \Default 'yes' if not the \kw{simulation} command is used.
 \elist

\subsubsection{\hbracket{facies-probabilities}}  \newkw{facies-probabilities}
 \slist
   \item \Description Triggers generation of facies probabilities. Absolute: facies probabilities are generated based on inverted parameters including background model. Relative: Background model is not used when facies probabilities are generated.
   \item \Argument 'absolute' or 'relative'
   \item \Default
 \elist


\subsection{\hbracket{estimation-settings}} \newkw{estimation-settings}
 \slist
   \item \Description Only valid with the \kw{mode} 'estimation'. Controls what will be estimated. Note that these commands can only turn off estimations - a parameter that is given will not be estimated even if it says so here.
   \item \Argument Elements for different estimation settings.
   \item \Default
 \elist

\subsubsection{\hbracket{estimate-background}}  \newkw{estimate-background}
 \slist
   \item \Description If no, background will not be estimated unless needed for other simulation.
   \item \Argument 'yes' or 'no'
   \item \Default
 \elist

\subsubsection{\hbracket{estimate-correlations}}  \newkw{estimate-correlations}
 \slist
   \item \Description If no, correlations will not be estimated unless needed for other simulation.
   \item \Argument 'yes' or 'no'
   \item \Default
 \elist

\subsubsection{\hbracket{estimate-wavelet-or-noise}}  \newkw{estimate-wavelet-or-noise}
 \slist
   \item \Description If no, wavelets and/or noise will not be estimated unless needed for other simulation.
   \item \Argument 'yes' or 'no'
   \item \Default
 \elist

 \section{\hbracket{project-settings}\necessary} \newkw{project-settings}
 \slist
   \item \Description Controls inversion volume, output and advanced program settings.
   \item \Argument
   \item \Default
 \elist

\subsection{\hbracket{output-volume}\necessary} \newkw{output-volume}
 \slist
   \item \Description Defines the core inversion volume. All grid output will be given in this volume.
   \item \Argument
   \item \Default
 \elist

\subsubsection{\hbracket{interval-two-surfaces}} \newkw{interval-two-surfaces}
 \slist
   \item \Description This or \kw{interval-one-surface} must be given. One way to give the top and bottom limitations. Must be used if output in depth domain is desired. Alternative is \kw{interval-one-surface}.
   \item \Argument
   \item \Default
 \elist

\paragraph{\hbracket{top-surface}\necessary} \newkw{top-surface}
 \slist
   \item \Description File name(s) for top surface file(s).
   \item \Argument
   \item \Default
 \elist

\paragraph{\hbracket{time-file}\necessary} \newkw{time-file}
 \slist
   \item \Description
   \item \Argument
   \item \Default
 \elist

\paragraph{\hbracket{base-surface}\necessary} \newkw{base-surface}
 \slist
   \item \Description File name(s) for base surface file(s).
   \item \Argument
   \item \Default
 \elist

\paragraph{\hbracket{number-of-layers}} \newkw{number-of-layers}
 \slist
   \item \Description Integer value giving how many layers to use between top and base surface.
   \item \Argument Integer
   \item \Default
 \elist

\subsubsection{\hbracket{interval-one-surface}} \newkw{interval-one-surface}
 \slist
   \item \Description This or \kw{interval-two-surfaces} must be given. Using this command gives parallel top and base of inversion interval.
   \item \Argument
   \item \Default
 \elist

\subsection{\hbracket{io-settings}} \newkw{io-settings}
 \slist
   \item \Description Holds commands that deal with what output to give and where, and where to find input.
   \item \Argument
   \item \Default
 \elist

\subsection{\hbracket{advanced-settings}} \newkw{advanced-settings}
 \slist
   \item \Description A collection of different commands that control advanced aspects of the program control.
   \item \Argument
   \item \Default
 \elist
