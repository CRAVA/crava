\chapter{Release notes}\index{release notes}
The tags, e.g. \jira{49}, are links to the \crava project management
system called
\href{http://www.atlassian.com/software/jira/}{JIRA}.
\index{JIRA, project management} Note that access permission is
required to be able to open the links.

\section{Changes from v1.2 to 2.0}

\begin{description}
\item [New features:] \mbox{ }
  \begin{itemize}
    \item Rock physics models in CRAVA
      \subitem Tabulated model
      \subitem Bounding model
    \item User inperface for rock physics models
    \item Trends in rock physics variables
    \item Distributions with trends
      \subitem Normal
      \subitem Beta
      \subitem Beta with end mass
    \item Generate background model from rock physics
    \item Calculate parameter covariance from rock physics
    \item Estimate facies probabilities using synthetic wells from rock physics models
    \item Smoothing values from illumination vector grid. \jira{162}
    \item Create multizone background model. \jira{317},  \jira{319}, \jira{325}
    \item Include beta distribution in nrlib/random \jira{331}
 \end{itemize}

\item [Changes:] \mbox{ }
  \begin{itemize}
   \item Check that facies probabilities are between 0 and 1. \jira{347}
   \item Adjust estimated wavelet such that it is smooth and the amplitude goes smoothly to zero in frequency domain. \jira{315}
   \item Make CRAVA compile under g++ 4.6.3 (Ubuntu 12.04). \jira{345}
 \end{itemize}

\item [Bug fixes:] \mbox{ }
  \begin{itemize}
  \item If ASCII grid output is requested, STORM grids cannot be imported to RMS \jira{224}
    \item Incorrect guard zone check done before smoothing seismic data in new resampling. \jira{330}
    \item Problems if <seg-start-time> is used when it should not have been. \jira{346}
    \item Failed to open temporal seismic file in crava format for writing. \jira{349}
    \item In new resampling algorithm of seismic data, wrong weights are used when doing interpolation. \jira{350}
    \item CRAVA incorrectly reports that no optimal global wavelet scale has been found. \jira{351}
    \item Some segy files are given the same path twice. \jira{352}
  \end{itemize}
\end{description}

\section{Changes from v1.1 to v1.2}

\begin{description}

\item [New features:] \mbox{ }
  \begin{itemize}
    \item Introduce new and more efficient resampling of seismic
      data. \jira{274}
    \item Add SI (=Vs*Rho) as possible input type for background
      model. \jira{287}
    \item For forward modelling allow earth model to be input as
      (AI,SI,Rho) and (AI,Vp/Vs,Rho). \jira{302}
    \item Add new SegY format "SIPX" to list of auto-detected
      formats.\jira{290}
    \item Added 3D wavelet estimation/inversion. \jira{304}, \jira{308},
      \jira{309}
  \end{itemize}

\item [Changes:] \mbox{ }
  \begin{itemize}
    \item Require seismic data in target zone+guard zone rather than
      in full inversion grid (simbox). \jira{216}
    \item Cleaned test suite and findgrammar for <TAB> and trailing
      blanks, and added "no trailings blanks" as requirement for passing
      test suite. \jira{275}, \jira{276}
    \item Use one wavelet (200ms) as default when estimating vertical
      padding size. Previous default was half a wavelet
      (100ms). \jira{300}
    \item Ensure that we log the parameter values of all parameters
      available under the \kw{advanced-settings} keyword. \jira{307}
  \end{itemize}

\item [Bug fixes:] \mbox{ }
  \begin{itemize}
    \item Fixed problem identifying the lateral geometry of seismic
      data. \jira{277}
    \item Vp and Vs were reported as input data types for background
      model in logFile.txt when AI and Vp/Vs were used. \jira{288}
    \item An incorrect grid size estimate made inversion runs stop
      even for medium sized grids. \jira{292}
    \item Fixed crash when the \kw{filter-elastic-logs} option was
      used. \jira{310}
    \item Removed memory leaks in test suite using Purify. \jira{265}.
  \end{itemize}
\end{description}


\section{Changes from v1.0 to v1.1}

\begin{description}

\item [New features:] \mbox{ }
  \begin{itemize}
    \item Allow background model to be specified with AI and\!/\!or
      \vp/\vs, as a complement to \vp, \vs, and $\rho$. Specify files
      for these parameters with keyword \kw{ai-file} and
      \kw{vp-vs-ratio-file}. \jira{214}
    \item Allow users to specify \vp/\vs for use in the
      reflection matrix.  Use keyword \kw{vp-vs-ratio}. \jira{219}
    \item Remove certain QC checks when \crava is run from
      \rms, as \rms will perform these tests. Activated with keyword
      \kw{rms-panel-mode}. \jira{221}
    \item When seismic data is given in SegY format, write the IL-XL
      range of data to the log file. \jira{230}, \jira{267}
    \item Create synthetic residuals which add up to original seismic
      when added to synthetic seismic. Activated with
      \kw{synthetic-residuals}. \jira{234}
    \item If the \kw{rms-panel-mode} mode has been requested and
      facies probabilities are not requested, well logs will not be
      (multi-parameter) filtered. \jira{242}
    \item Allow \vp/\vs to be estimated from wells. Activated
      with keyword \kw{vp-vs-ratio-from-wells}. If
      \kw{wavelet-estimation-interval} has been specified, this
      interval will also be used for the \vp/\vs estimate. \jira{247}
    \item If automatic SegY format detection fails, list trace
      header locations of known formats as well as their associated
      \crava-names. \jira{259}
    \item Give a warning if prior or posterior parameter correlations
      exceed one. To avoid this situation completely, some recoding is
      required. This is left for a later release (see
      \jira{257}). \jira{261}
    \item When specifying an area as UTM coordinates or surface,
      choose as inversion area the smallest IL-XL box that
      encloses the specified area. Activated with keyword
      \kw{snap-to-seismic-data}. \jira{266}
    \item Allow (some) traces to have bogus IL-XL information in
      header. If encountered, the IL-XL values are set to
      undefined. \jira{269}
    \item Write IL-XL values for well start positions whenever
      possible. This makes it easy to do inversion around a given
      well. \jira{270}
 \end{itemize}

\item [Changes:] \mbox{ }
  \begin{itemize}
    \item When the prior correlation between \vp and \vs cannot be
      estimated from well data it is set to $1/\sqrt{2}\approx
      0.7$. \jira{220}
    \item Made blocking of facies deterministic. If a cell has the
      same facies count for $n$ different facies, the cell value
      becomes facies\_to\_choose $= (k + 1)\ \%\ n$, where $k$ is the
      layer number. This used to be randomly chosen, so that identical
      runs could give different facies predictions. \jira{249}
    \item Changed the computation of SegY geometry, and improved error messages. The new version should be more accurate and robust.
  \end{itemize}

\item [Bug fixes:] \mbox{ }
  \begin{itemize}
    \item Fixed crash in background modelling when there were
      (unrealistically) few layers in grid. \jira{200}
    \item Background volumes of AI and\!/\!or \vp/\vs containing undefined
      values caused troubles. The fix is to set such undefined values
      to global mean. \jira{222}
    \item Removed bugs in estimation of wavelet norm, which indirectly
      led to ringing in inversion results. \jira{223}
    \item Number of allocated grids were incorrectly counted when
      estimating total memory requirement for \crava. \jira{251}
    \item Fixed crash when interval was specified with option
      \kw{interval-one-surface}. Wavelets had not been transformed
      prior to constant thickness inversion. Also, a wavelet polarity
      bug was fixed. New test cases 16 and 17 were added to avoid
      future constant thickness and wavelet polarity related
      bugs. \jira{252}, \jira{253}
    \item For short wells, deviation angles were incorrectly estimated
      due to an 10ms sample requirement. Also, the angle was
      calculated between well start and well end instead of for
      trajectory tangents.
    \item Fixed crash when project directory could not be
      created. This was a bug in the throw/catch system of
      NRLib. \jira{256}
    \item Fixed several bugs encountered when using 3D prior facies
      probability volumes. \jira{272}
    \item Fixed bug with interpolation of Storm-grids used for input.
 \end{itemize}
\end{description}

