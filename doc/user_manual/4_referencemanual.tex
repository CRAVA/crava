
\chapter{Model file reference manual}\index{reference manual, CRAVA model file elements}
\label{ap:model-file-reference}
\index{CRAVA model file@Crava model file!reference manual for elements}
The numbering shows the command grouping. A command with no sub-numbering expects a value to be given, otherwise, it is only a grouping of other commands.

File names are currently given with a path relative to the directory settings in <project-settings>-<io-settings>-<input/output/top-directory>. If these are not given, the path will always be relative to the working directory.

All commands are optional, unless otherwise stated. A necessary command under an optional is only necessary if the optional is given.

Standard grid formats for surfaces and 3D grids are given in \autoref{sec:gridformats}.

%%%%%%%%%%%%%%%%%%%%%%%%%%%%%%%%%%%%%%%%%%%%%%%%%%%%%%%%%%%%%%%%%%%%%%%%%
%%%%%                            ACTIONS                            %%%%%
%%%%%%%%%%%%%%%%%%%%%%%%%%%%%%%%%%%%%%%%%%%%%%%%%%%%%%%%%%%%%%%%%%%%%%%%%

\section{\hbracket{actions}\necessary} \newkw{actions}
 \slist
   \item \Description Controls the main purpose of the run.
   \item \Argument Elements specifying the main purpose
 \elist

\subsection{\hbracket{mode}\necessary}  \newkw{mode}
 \slist
   \item \Description Inversion:\index{inversion!mode} Invert seismic input data to elastic parameters and/or facies probabilities. Needs seismic data and volume, all other missing data will be estimated.
Forward:\index{forward!mode} Create seismic response from background model. Not able to estimate anything.
Estimation:\index{estimation!mode} Checks input data and performs estimation of lacking information for inversion, but stops before inversion.
   \item \Argument 'inversion', 'forward' or 'estimation'
 \elist

\subsection{\hbracket{inversion-settings}}  \newkw{inversion-settings}
 \slist
   \item \Description Controls aspects of the inversion. Only valid with the \kw{mode} 'inversion' above.
   \item \Argument Elements controlling the inversion
   \item \Default
 \elist

\subsubsection{\hbracket{prediction}}  \newkw{prediction}
 \slist
   \item \Description Controls whether predicted elastic parameters will be generated.
   \item \Argument 'yes' or 'no'
   \item \Default
 \elist

\subsubsection{\hbracket{simulation}}  \newkw{simulation}
 \slist
   \item \Description Controls aspects of the simulation of elastic parameters.
   \item \Argument Elements controlling the simulation of elastic parameters
 \elist



\paragraph{\hbracket{seed}}  \newkw{seed}
 \slist
   \item \Description A number used to initialise the random generator. Running a model file with a given seed will give the same simulation results each time.
   \item \Argument Integer
   \item \Default 0
 \elist

\paragraph{\hbracket{seed-file}}  \newkw{seed-file}
 \slist
   \item \Description Alternative to \kw{seed}. This is an ASCII file containing a number. At the termination of the run, the file will be overwritten with a seed generated by the random generator. Thus, a model file using this will generate different simulation results on sequential runs.
   \item \Argument ASCII file containing a number
   \item \Default
 \elist

\paragraph{\hbracket{number-of-simulations}}  \newkw{number-of-simulations}
 \slist
   \item \Description Integer value giving the number of stochastic realizations to generate.
   \item \Argument Integer
   \item \Default 0
 \elist

\subsubsection{\hbracket{kriging-to-wells}}  \newkw{kriging-to-wells}
 \slist
   \item \Description Should the realizations be kriged to well data?
   \item \Argument 'yes' or 'no'
   \item \Default 'yes' if not the \kw{simulation} command is used.
 \elist

\subsubsection{\hbracket{facies-probabilities}}  \newkw{facies-probabilities}
 \slist
   \item \Description Should facies probabilities be estimated?
   \item \Argument 'yes' or 'no'
   \item \Default 'no'
 \elist




\subsection{\hbracket{estimation-settings}} \newkw{estimation-settings}
 \slist
   \item \Description Controls what will be estimated. Only valid with the \kw{mode} 'estimation'. Note that these commands can only turn off estimations - a parameter that is given will not be estimated even if it says so here.
   \item \Argument Elements controlling what to estimate
   \item \Default
 \elist

\subsubsection{\hbracket{estimate-background}}  \newkw{estimate-background}
 \slist
   \item \Description If 'no', background will not be estimated unless needed for other estimation.
   \item \Argument 'yes' or 'no'
   \item \Default 'yes'
 \elist

\subsubsection{\hbracket{estimate-correlations}}  \newkw{estimate-correlations}
 \slist
   \item \Description If 'no', correlations will not be estimated unless needed for other estimation.
   \item \Argument 'yes' or 'no'
   \item \Default 'yes'
 \elist

\subsubsection{\hbracket{estimate-wavelet-or-noise}}  \newkw{estimate-wavelet-or-noise}
 \slist
   \item \Description If 'no', wavelets and/or noise will not be estimated unless needed for other estimation.
   \item \Argument 'yes' or 'no'
   \item \Default 'yes'
 \elist

%%%%%%%%%%%%%%%%%%%%%%%%%%%%%%%%%%%%%%%%%%%%%%%%%%%%%%%%%%%%%%%%%%%%%%%%%
%%%%%                          PROJECT SETTINGS                     %%%%%
%%%%%%%%%%%%%%%%%%%%%%%%%%%%%%%%%%%%%%%%%%%%%%%%%%%%%%%%%%%%%%%%%%%%%%%%%

 \section{\hbracket{project-settings}\necessary} \newkw{project-settings}
 \slist
   \item \Description Controls inversion volume, output and advanced program settings.
   \item \Argument Elements controlling inversion volume, output and advanced program settings
   \item \Default
 \elist

\subsection{\hbracket{output-volume}\necessary} \newkw{output-volume}
 \slist
   \item \Description Defines the core inversion volume. All grid output will be given in this volume.
   \item \Argument Elements defining the core inversion volume.
   \item \Default
 \elist

\subsubsection{\hbracket{interval-two-surfaces}} \newkw{interval-two-surfaces}
 \slist
   \item \Description One way to give the top and bottom limitations. Must be used if output in depth domain is desired. This or \kw{interval-one-surface} must be given.
   \item \Argument
   \item \Default
 \elist

\paragraph{\hbracket{top-surface}\necessary} \newkw{top-surface}
 \slist
   \item \Description File name(s) for top surface file(s).
   \item \Argument Elements controlling the top surface
   \item \Default
 \elist

\subparagraph{\hbracket{time-file}} \newkw{time-file}
 \slist
   \item \Description File name for standard surface file giving top surface in time. This or \kw{time-value} must be given.
   \item \Argument File name
   \item \Default
 \elist

\subparagraph{\hbracket{time-value}} \newkw{time-value}
 \slist
   \item \Description Value giving the top time for the inversion interval. This or \kw{time-file} must be given.
   \item \Argument Value
   \item \Default
 \elist

 \subparagraph{\hbracket{depth-file}} \newkw{depth-file}
 \slist
   \item \Description File name for standard surface file giving top surface in depth.
   \item \Argument File name
   \item \Default
 \elist

\paragraph{\hbracket{base-surface}\necessary} \newkw{base-surface}
 \slist
   \item \Description File name(s) for base surface file(s).
   \item \Argument Elements controlling the base surface
   \item \Default
 \elist

\subparagraph{\hbracket{time-file}} \rnewkw{time-file}{time-file2}
\slist
   \item \Description File name for standard surface file giving base surface in time. This or \rkw{time-value}{time-value2} must be given.
   \item \Argument File name
   \item \Default
\elist

\subparagraph{\hbracket{time-value}} \rnewkw{time-value}{time-value2}
\slist
   \item \Description Value giving the base time for the inversion interval. This or \rkw{time-file}{time-file2} must be given.
   \item \Argument Value
   \item \Default
\elist

\subparagraph{\hbracket{depth-file}} \rnewkw{depth-file}{depth-file2}
\slist
   \item \Description File name for standard surface file giving base surface in depth.
   \item \Argument File name
   \item \Default
\elist


\paragraph{\hbracket{number-of-layers}} \newkw{number-of-layers}
 \slist
   \item \Description Integer value giving how many layers to use between top and base surface.
   \item \Argument Integer
   \item \Default
 \elist

\paragraph{\hbracket{velocity-field}} \newkw{velocity-field}
 \slist
   \item \Description File name for standard 3D grid file. Gives more detailed depth conversion information. Without this, constant velocity per trace is used. If only one depth surface is given, this is used to compute the other. Otherwise, the depth interval will always match both surfaces, but the velocity field is scaled and used for internal depth computations. Can not be used with \kw{velocity-field-from-inversion}.
   \item \Argument File name
   \item \Default
 \elist

\paragraph{\hbracket{velocity-field-from-inversion}} \newkw{velocity-field-from-inversion}
 \slist
   \item \Description If given, velocity field from inversion is used for depth conversion. See \kw{velocity-field} for details on how this is done. Can not be used with \kw{velocity-field}.
   \item \Argument 'yes' or 'no'
   \item \Default
 \elist

\subsubsection{\hbracket{interval-one-surface}} \newkw{interval-one-surface}
 \slist
   \item \Description Using this command gives parallel top and base of inversion interval. This or \kw{interval-two-surfaces} must be given.
   \item \Argument Elements for parallel top and base inversion interval
   \item \Default
 \elist

\paragraph{\hbracket{reference-surface}} \newkw{reference-surface}
 \slist
   \item \Description File name for standard surface file. The top and base surfaces for the inversion interval will be parallel to this.
   \item \Argument File name
   \item \Default
 \elist

\paragraph{\hbracket{shift-to-interval-top}} \newkw{shift-to-interval-top}
 \slist
   \item \Description Value giving the distance from reference surface to top surface. This value is added to the reference surface to create the top surface.
   \item \Argument Value
   \item \Default
 \elist

\paragraph{\hbracket{thickness}}\newkw{thickness}
 \slist
   \item \Description Value giving the thickness of the inversion interval. This value is added to the top surface to create the base surface.
   \item \Argument Value
   \item \Default
 \elist

\paragraph{\hbracket{sample-density}}\newkw{sample-density}
 \slist
   \item \Description Value giving the thickness of a layer in the inversion interval. The thickness should be divisible by this value.
   \item \Argument Value
   \item \Default
 \elist


\subsubsection{\hbracket{area-from-surface}}\newkw{area-from-surface}
 \slist
   \item \Description Inversion area can be defined by a surface. Then
     the name of the surface is given in this command. Other ways to
     define inversion area are by the commands \kw{utm-coordinates} or
     \kw{inline-crossline-numbers}. If none of these commands are
     used, the area is defined by the first seismic data file, or from
     Vp if we do forward modelling.
   \item \Argument
   \item \Default
 \elist

\paragraph{\hbracket{file-name}}\rnewkw{file-name}{file-name5}
\slist
   \item \Description File name for standard surface file.
   \item \Argument File name
   \item \Default
 \elist

\paragraph{\hbracket{snap-to-seismic-data}}\newkw{snap-to-seismic-data}
\slist
   \item \Description Find the smallest rectangular IL-XL box
     enclosing the entire surface and do the inversion using these
     IL-XL values. This allows a user to specify an inversion area in
     UTM and get the inversion volume aligned with seismic data.
   \item \Argument  'yes' or 'no'
   \item \Default 'no'
 \elist


\subsubsection{\hbracket{utm-coordinates}}\newkw{utm-coordinates}
 \slist
   \item \Description Describe area by UTM coordinates.
   \item \Argument
   \item \Default
 \elist

\paragraph{\hbracket{reference-point-x}}\newkw{reference-point-x}
 \slist
   \item \Description Value giving the x-coordinate of a corner of the area.
   \item \Argument Value
   \item \Default
 \elist

\paragraph{\hbracket{reference-point-y}}\newkw{reference-point-y}
 \slist
   \item \Description Value giving the y-coordinate of a corner of the area.
   \item \Argument Value
   \item \Default
 \elist

\paragraph{\hbracket{length-x}}\newkw{length-x}
 \slist
   \item \Description Value giving the area length along the rotated x-axis.
   \item \Argument Value
   \item \Default
 \elist

\paragraph{\hbracket{length-y}}\newkw{length-y}
 \slist
   \item \Description Value giving the area length along the rotated y-axis.
   \item \Argument Value
   \item \Default
 \elist

\paragraph{\hbracket{sample-density-x}}\newkw{sample-density-x}
 \slist
   \item \Description Cell size along the rotated x-axis.
   \item \Argument Integer
   \item \Default
 \elist

\paragraph{\hbracket{sample-density-y}}\newkw{sample-density-y}
 \slist
   \item \Description Cell size along the rotated y-axis.
   \item \Argument Integer
   \item \Default
 \elist

\paragraph{\hbracket{angle}}\newkw{angle}
 \slist
   \item \Description Orientation of the azimuth.
   \item \Argument
   \item \Default
 \elist

\paragraph{\hbracket{snap-to-seismic-data}}\rnewkw{snap-to-seismic-data}{snap-to-seismic-data2}
\slist
   \item \Description Find the smallest rectangular IL-XL box
     enclosing the entire UTM specified area and do the inversion
     using these IL-XL values. This allows a user to specify an
     inversion area in UTM coordinates and get an inversion volume
     aligned with seismic data. Keywords \hbracket{sample-density-x}
     and \hbracket{sample-density-y} are not needed when snapping
     is activated.
   \item \Argument  'yes' or 'no'
   \item \Default 'no'
 \elist


\subsubsection{\hbracket{inline-crossline-numbers}}\newkw{inline-crossline-numbers}
 \slist
   \item \Description Describe area by inline and crossline numbers. il-start and xl-start must be given if this command is used, the other variables are optional. The numbers which are not specified are taken from the SegY file containing seismic data. The command is only working if seismic data are given on SegY format.
   \item \Argument
   \item \Default
 \elist

\paragraph{\hbracket{il-start}}\newkw{il-start}
 \slist
   \item \Description Start value for inline.
   \item \Argument
   \item \Default
 \elist
\paragraph{\hbracket{il-end}}\newkw{il-end}
 \slist
   \item \Description End value for inline.
   \item \Argument
   \item \Default
 \elist
\paragraph{\hbracket{xl-start}}\newkw{xl-start}
 \slist
   \item \Description Start value for crossline.
   \item \Argument
   \item \Default
 \elist
\paragraph{\hbracket{xl-end}}\newkw{xl-end}
 \slist
   \item \Description End value for crossline.
   \item \Argument
   \item \Default
 \elist
\paragraph{\hbracket{il-step}}\newkw{il-step}
 \slist
   \item \Description Step value for inline.
   \item \Argument
   \item \Default
 \elist
\paragraph{\hbracket{xl-step}}\newkw{xl-step}
 \slist
   \item \Description Step value for crossline.
   \item \Argument
   \item \Default
 \elist

\subsection{\hbracket{time-to-depth-mapping-for-3d-wavelet}} \newkw{time-to-depth-mapping-for-3d-wavelet}
 \slist
   \item \Description Defines the mapping between pseudo-depth and local time in the target area for 3D wavelet.
   \item \Argument Reference depth, velocity and time surface for mapping
   \item \Default
 \elist

\subsubsection{\hbracket{reference-depth}} \newkw{reference-depth}
 \slist
   \item \Description Holds the z-value for the reference depth for target area
   \item \Argument Depth in meter
   \item \Default
\elist

\subsubsection{\hbracket{average-velocity}} \newkw{average-velocity}
 \slist
   \item \Description Holds the average velocity in the target area
   \item \Argument Velocity in meter/second
   \item \Default
\elist

\subsubsection{\hbracket{reference-time-surface}} \newkw{reference-time-surface}
 \slist
   \item \Description File name for the time surface corresponding to the reference depth. Standard surface format.
   \item \Argument File name
   \item \Default
\elist

\subsection{\hbracket{io-settings}} \newkw{io-settings}
 \slist
   \item \Description Holds commands that deal with what output to give and where, and where to find input.
   \item \Argument Elements controlling output and input
   \item \Default
 \elist

\subsubsection{\hbracket{top-directory}} \newkw{top-directory}
 \slist
   \item \Description Directory name giving the working directory for the model file. Must end with directory separator.
   \item \Argument Directory name
   \item \Default
 \elist

\subsubsection{\hbracket{input-directory}} \newkw{input-directory}
 \slist
   \item \Description Directory name, relative to \kw{top-directory}, for root directory for input files. Must end with directory separator.
   \item \Argument Directory name
   \item \Default
 \elist

\subsubsection{\hbracket{output-directory}} \newkw{output-directory}
 \slist
   \item \Description Directory name, relative to \kw{top-directory}, for root directory for output files. Must end with directory separator.
   \item \Argument Directory name
   \item \Default
 \elist


\subsubsection{\hbracket{grid-output}}\newkw{grid-output}
 \slist
   \item \Description All commands related to output given as grids are gathered here.
   \item \Argument Elements controlling output given as grids
   \item \Default
 \elist

\paragraph{\hbracket{domain}}\newkw{domain}
 \slist
   \item \Description Commands specifying which domain output should be in.
   \item \Argument Elements controlling the output domain
   \item \Default
 \elist

\subparagraph{\hbracket{depth}}\newkw{depth}
 \slist
   \item \Description Should output come in depth domain? Requires information under \kw{interval-two-surfaces}.
   \item \Argument 'yes' or 'no'
   \item \Default 'no'
 \elist

\subparagraph{\hbracket{time}}\newkw{time}
 \slist
   \item \Description Should output come in time domain?
   \item \Argument 'yes' or 'no'
   \item \Default 'yes'
 \elist

\paragraph{\hbracket{format}}\newkw{format}
 \slist
   \item \Description Control of the format of output grids.
   \item \Argument Elements controlling the format of output grids
   \item \Default
 \elist

\subparagraph{\hbracket{segy-format}}\newkw{segy-format}
 \slist
   \item \Description Information about the segy format. By default CRAVA recognises SeisWorks, IESX, SIP and Charisma, see Table~\ref{tab:segyformats}.
   \item \Argument Elements containing information about the segy format.
   \item \Default
 \elist

\subsubparagraph{\hbracket{standard-format}}\newkw{standard-format}
 \slist
   \item \Description Giving the starting format for modifications.
   \item \Argument 'seisworks', 'iesx', 'charisma' or 'SIP'
   \item \Default 'seisworks'
 \elist

\subsubparagraph{\hbracket{location-x}}\newkw{location-x}
 \slist
   \item \Description The byte location for the x-coordinate in the trace header.
   \item \Argument Integer
   \item \Default
 \elist

\subsubparagraph{\hbracket{location-y}}\newkw{location-y}
 \slist
   \item \Description The byte location for the y-coordinate in the trace header.
   \item \Argument Integer
   \item \Default
 \elist

\subsubparagraph{\hbracket{location-il}}\newkw{location-il}
 \slist
   \item \Description The byte location for the inline in the trace header.
   \item \Argument Integer
   \item \Default
 \elist

\subsubparagraph{\hbracket{location-xl}}\newkw{location-xl}
 \slist
   \item \Description The byte location for the crossline in the trace header.
   \item \Argument Integer
   \item \Default
 \elist

\subsubparagraph{\hbracket{bypass-coordinate-scaling}}\newkw{bypass-coordinate-scaling}
 \slist
   \item \Description Indicates whether coordinate scaling information should be used.
   \item \Argument 'yes' or 'no'
   \item \Default
 \elist

\subsubparagraph{\hbracket{location-scaling-coefficient}}\newkw{location-scaling-coefficient}
 \slist
   \item \Description
   \item \Argument Integer
   \item \Default
 \elist

\subparagraph{\hbracket{segy}}\newkw{segy}
 \slist
   \item \Description Should grid output come as segy?
   \item \Argument 'yes' or 'no'
   \item \Default
 \elist

\subparagraph{\hbracket{storm}}\newkw{storm}
 \slist
   \item \Description Should grid output come as storm?
   \item \Argument 'yes' or 'no'
   \item \Default 'yes' if \kw{format} command is not given.
 \elist


\subparagraph{\hbracket{crava}}\newkw{crava}
 \slist
   \item \Description Should grid output come in crava binary format?
   \item \Argument 'yes' or 'no'
   \item \Default
 \elist

\subparagraph{\hbracket{sgri}}\newkw{sgri}
 \slist
   \item \Description Should grid output come as storm sgri?
   \item \Argument 'yes' or 'no'
   \item \Default
 \elist

\subparagraph{\hbracket{ascii}}\newkw{ascii}
 \slist
   \item \Description Should grid output come as storm ascii?
   \item \Argument 'yes' or 'no'
   \item \Default
 \elist

\paragraph{\hbracket{elastic-parameters}}\newkw{elastic-parameters}
 \slist
   \item \Description Controls which elastic grid parameters to output. All are 'yes' or 'no'.
   \item \Argument Elements controlling which elastic parameters to output
   \item \Default If this command is not given, vp, vs and density are written.
 \elist

\subparagraph{\hbracket{vp}}\newkw{vp}
 \slist
   \item \Description
   \item \Argument 'yes' or 'no'
   \item \Default
 \elist

\subparagraph{\hbracket{vs}}\newkw{vs}
 \slist
   \item \Description
   \item \Argument 'yes' or 'no'
   \item \Default
 \elist

\subparagraph{\hbracket{density}}\newkw{density}
 \slist
   \item \Description
   \item \Argument 'yes' or 'no'
   \item \Default
 \elist

\subparagraph{\hbracket{lame-lambda}}\newkw{lame-lambda}
 \slist
   \item \Description
   \item \Argument 'yes' or 'no'
   \item \Default
 \elist


\subparagraph{\hbracket{lame-mu}}\newkw{lame-mu}
 \slist
   \item \Description
   \item \Argument 'yes' or 'no'
   \item \Default
 \elist

\subparagraph{\hbracket{poisson-ratio}}\newkw{poisson-ratio}
 \slist
   \item \Description
   \item \Argument 'yes' or 'no'
   \item \Default
 \elist

\subparagraph{\hbracket{ai}}\newkw{ai}
 \slist
   \item \Description
   \item \Argument 'yes' or 'no'
   \item \Default
 \elist

\subparagraph{\hbracket{si}}\newkw{si}
 \slist
   \item \Description
   \item \Argument 'yes' or 'no'
   \item \Default
 \elist

\subparagraph{\hbracket{vp-vs-ratio}}\newkw{vp-vs-ratio}
 \slist
   \item \Description
   \item \Argument 'yes' or 'no'
   \item \Default
 \elist

\subparagraph{\hbracket{murho}}\newkw{murho}
\slist
  \item \Description
  \item \Argument 'yes' or 'no'
  \item \Default
\elist

\subparagraph{\hbracket{lambdarho}}\newkw{lambdarho}
 \slist
   \item \Description
   \item \Argument 'yes' or 'no'
   \item \Default
\elist

\subparagraph{\hbracket{background}}\newkw{background}
 \slist
   \item \Description
   \item \Argument 'yes' or 'no'
   \item \Default
\elist

\subparagraph{\hbracket{background-trend}}\newkw{background-trend}
 \slist
   \item \Description
   \item \Argument 'yes' or 'no'
   \item \Default
\elist

\paragraph{\hbracket{seismic-data}}\newkw{seismic-data}
 \slist
   \item \Description Controls which seismic data parameters to output. All are 'yes' or 'no'.
   \item \Argument Elements controlling which seismic data to output
   \item \Default
 \elist
\subparagraph{\hbracket{original}}\newkw{original}
 \slist
   \item \Description Write original seismic data to file.
   \item \Argument 'yes' or 'no'
   \item \Default
 \elist

\subparagraph{\hbracket{synthetic}}\newkw{synthetic}
 \slist
   \item \Description Generate synthetic seismic from the inverted data.
   \item \Argument 'yes' or 'no'
   \item \Default
 \elist

\subparagraph{\hbracket{residuals}}\newkw{residuals}
 \slist
   \item \Description Residuals computed in the actual inversion. Will not add up to original seismic data when combined with synthetic seismic, due to some filtering and numerical imprecision.
   \item \Argument 'yes' or 'no'
   \item \Default 'no'
\elist

\subparagraph{\hbracket{synthetic-residuals}}\newkw{synthetic-residuals}
 \slist
   \item \Description Residuals computed by taking the original seismic and subtracting the synthetic seismic. This means that these parameters are a matched set.
   \item \Argument 'yes' or 'no'
   \item \Default 'no'
\elist

\paragraph{\hbracket{other-parameters}}\newkw{other-parameters}
 \slist
   \item \Description Controls which other parameters to output. All are 'yes' or 'no'.
   \item \Argument Elements controlling which other to output
   \item \Default
 \elist

\subparagraph{\hbracket{facies-probabilities}}\rnewkw{facies-probabilities}{facies-probabilities3}
 \slist
   \item \Description Write facies probabilities to file.
   \item \Argument 'yes' or 'no'
   \item \Default 'yes' if facies estimation is requested and \kw{facies-probabilities-with-undef} is not specified
 \elist

\subparagraph{\hbracket{facies-probabilities-with-undef}}\newkw{facies-probabilities-with-undef}
 \slist
   \item \Description Write facies probabilities with undefined value to file.
   \item \Argument 'yes' or 'no'
   \item \Default 'no'
 \elist

\subparagraph{\hbracket{facies-likelihood}}\newkw{facies-likelihood}
 \slist
   \item \Description Write likelihood for inverted seismic for each
     facies, $p(m|f)$.
   \item \Argument 'yes' or 'no'
   \item \Default 'no'
 \elist

\subparagraph{\hbracket{time-to-depth-velocity}}\newkw{time-to-depth-velocity}
 \slist
   \item \Description Write time-to-depth velocity to file.
   \item \Argument 'yes' or 'no'
   \item \Default
 \elist

\subparagraph{\hbracket{extra-grids}}\newkw{extra-grids}
 \slist
   \item \Description Temporary, will be replaced. Currently triggers writing of
   \begin{itemize}
   \item Estimated background files in extended versions (go above and below inversion volume).
   \item Estimated background files in standard volume.
   \end{itemize}
   \item \Argument 'yes' or 'no'
   \item \Default
\elist

\subparagraph{\hbracket{correlations}}\newkw{correlations}
 \slist
   \item \Description These are the posterior correlations between vp, vs and density after inversion.
   \item \Argument 'yes' or 'no'
   \item \Default
\elist

\subparagraph{\hbracket{seismic-quality-grid}}\newkw{seismic-quality-grid}
 \slist
   \item \Description Quality grid kriged from values for fit between facies probabilities and facies observed in each of the wells.
   \item \Argument 'yes' or 'no'
   \item \Default
\elist


\subsubsection{\hbracket{well-output}}\newkw{well-output}
 \slist
   \item \Description Collects all output that can be given in well format. Wells contain logs for vp, vs and density, each of these in four versions: Raw, filtered to background frequency, filtered to seismic frequency and seismic resolution. In addition, the facies log is written if found.
   \item \Argument Elements collecting output given in well format
   \item \Default
 \elist

\paragraph{\hbracket{format}}\rnewkw{format}{format2}
 \slist
   \item \Description Controls well formats used for output. Default is RMS.
   \item \Argument Elements controlling the formats used for output
   \item \Default
 \elist

\subparagraph{\hbracket{rms}}\newkw{rms}
 \slist
   \item \Description Controls if wells are written on RMS format.
   \item \Argument 'yes' or 'no'
   \item \Default
 \elist

 \subparagraph{\hbracket{norsar}}\newkw{norsar}
 \slist
   \item \Description Controls if wells are written on NORSAR format.
   \item \Argument 'yes' or 'no'
   \item \Default
 \elist

\paragraph{\hbracket{wells}}\newkw{wells}
 \slist
   \item \Description Writes wells following original sampling density.
   \item \Argument 'yes' or 'no'
   \item \Default
 \elist

\paragraph{\hbracket{blocked-wells}}\newkw{blocked-wells}
 \slist
   \item \Description Writes wells sampled to internal grid resolution.
   \item \Argument 'yes' or 'no'
   \item \Default
 \elist

\paragraph{\hbracket{blocked-logs}}\newkw{blocked-logs}
 \slist
   \item \Description Not currently active.
   \item \Argument
   \item \Default
 \elist

\subsubsection{\hbracket{wavelet-output}}\newkw{wavelet-output}
 \slist
   \item \Description Collects all output that can be given for wavelets.
   \item \Argument Elements controlling wavelet output
   \item \Default
 \elist

\paragraph{\hbracket{format}}\rnewkw{format}{format3}
 \slist
   \item \Description Controls wavelet formats used for output. Default is JASON.
   \item \Argument Elements controlling the formats used for output
   \item \Default
 \elist

\subparagraph{\hbracket{jason}}\newkw{jason}
 \slist
   \item \Description Controls if wavelets are written on JASON form, being 'wlt' format.
   \item \Argument 'yes' or 'no'
   \item \Default 'yes'
 \elist

 \subparagraph{\hbracket{norsar}}\rnewkw{norsar}{norsar2}
 \slist
   \item \Description Controls if wavelets are written on NORSAR form, being 'swav' format.
   \item \Argument 'yes' or 'no'
   \item \Default 'no'
 \elist

\paragraph{\hbracket{well-wavelets}}\newkw{well-wavelets}
 \slist
   \item \Description Writes estimated wavelets for each well used for wavelet estimation. Note: These wavelets are not optimally shifted, but aligned for easy comparison.
	\item \Argument 'yes' or 'no'
	\item \Default 'no'
 \elist

\paragraph{\hbracket{global-wavelets}}\newkw{global-wavelets}
 \slist
   \item \Description Writes global wavelets for each seismic angle.
	\item \Argument 'yes' or 'no'
	\item \Default 'no'
 \elist

\paragraph{\hbracket{local-wavelets}}\newkw{local-wavelets}
 \slist
   \item \Description Writes estimated local wavelet shift and scale surfaces. Can only be written when \kw{local-wavelet} is requested.
	\item \Argument 'yes' or 'no'
	\item \Default
 \elist

\subsubsection{\hbracket{other-output}}\newkw{other-output}
 \slist
   \item \Description Controls output that is neither standard grid nor well.
   \item \Argument Elements controlling output
   \item \Default
 \elist

\paragraph{\hbracket{extra-surfaces}}\newkw{extra-surfaces}
 \slist
   \item \Description Temporary, will be replaced. Currently writes:
   \begin{itemize}
   \item Top and base surface for constant thickness interval used for log filtering and facies probabilities.
   \item Top and base surface for extended inversion interval computed from correlation surface.
   \item Top and base surface for background estimation interval (larger than inversion interval).
   \end{itemize}
   \item \Argument 'yes' or 'no'
   \item \Default
 \elist

\paragraph{\hbracket{prior-correlations}}\newkw{prior-correlations}
 \slist
   \item \Description Write prior correlation files.
   \item \Argument 'yes' or 'no'
   \item \Default
 \elist

\paragraph{\hbracket{background-trend-1d}}\newkw{background-trend-1d}
 \slist
   \item \Description Write the background trend as 1D curve.
   \item \Argument 'yes' or 'no'
   \item \Default
 \elist

\paragraph{\hbracket{local-noise}}\newkw{local-noise}
 \slist
   \item \Description Writes estimated local noise surface. Can only be written when \kw{local-noise-scaled} or \kw{estimate-local-noise} is requested.
   \item \Argument 'yes' or 'no'
   \item \Default
 \elist

\paragraph{\hbracket{rock-physics-distributions}}\newkw{rock-physics-distributions}
 \slist
   \item \Description Writes rock physics distribution per facies, in a vp, vs, and density STORM-grid. Only available when facies probabilities are computed.
   \item \Argument 'yes' or 'no'
   \item \Default 'no'
\elist

\paragraph{\hbracket{error-file}}\newkw{error-file}
 \slist
   \item \Description Writes all errors to a separate file, in
   addition to the log file.
   \item \Argument 'yes' or 'no'
   \item \Default 'no'
\elist

\paragraph{\hbracket{task-file}}\newkw{task-file}
 \slist
   \item \Description Writes all tasks to a separate file, in
   addition to the log file.
   \item \Argument 'yes' or 'no'
   \item \Default 'no'
\elist

\subsubsection{\hbracket{file-output-prefix}}\newkw{file-output-prefix}
 \slist
   \item \Description Common prefix added to all files written in the run. Identifies the run.
   \item \Argument String
   \item \Default
 \elist

\subsubsection{\hbracket{log-level}}\newkw{log-level}
 \slist
   \item \Description
   \item \Argument String. Possible values are error, warning, low, medium, high.
   \item \Default Low
 \elist

\subsection{\hbracket{advanced-settings}} \newkw{advanced-settings}
 \slist
   \item \Description A collection of different commands that control advanced aspects of the program control.
   \item \Argument Commands controlling advanced aspects of the program control
   \item \Default
 \elist

\subsubsection{\hbracket{fft-grid-padding}}\newkw{fft-grid-padding}
 \slist
   \item \Description Controls the padding size, can be used to optimize memory or improve visual results. Padding should be at least one range laterally, and a wavelet length vertically to avoid edge effects.
   \item \Argument Elements controlling the padding size
   \item \Default
 \elist

\paragraph{\hbracket{x-fraction}}\newkw{x-fraction}
 \slist
   \item \Description Value telling how large the padding in the x-direction should be relative to the x-length. %If the padding command is not called, proper padding will be estimated.
   \item \Argument Value
   \item \Default 0.0
 \elist

\paragraph{\hbracket{y-fraction}}\newkw{y-fraction}
 \slist
   \item \Description Value telling how large the padding in the x-direction should be relative to the y-length.
   \item \Argument Value
   \item \Default 0.0
 \elist

\paragraph{\hbracket{z-fraction}}\newkw{z-fraction}
 \slist
   \item \Description Value telling how large the padding in the x-direction should be relative to the thickness.
   \item \Argument Value
   \item \Default 0.0
 \elist

\subsubsection{\hbracket{use-intermediate-disk-storage}} \newkw{use-intermediate-disk-storage}
 \slist
   \item \Description When running under Windows with less physical
     memory than the program requires, this activates a built-in
     smart-swap. It is more efficient to use this smart-swapping than
     the built-in windows paging system. Linux/Unix swap is so
     efficient that this option has little effect there. If you run
     Crava on a machine that you share with other users, it can be
     wise to use this if you know that Crava will need most of the
     memory.
   \item \Argument 'yes' or 'no'
   \item \Default
 \elist

\subsubsection{\hbracket{vp-vs-ratio}}\rnewkw{vp-vs-ratio}{vp-vs-ratio2}
 \slist
   \item \Description Value of Vp/Vs ratio used in reflection
     matrix. By default, the Vp/Vs ratio is estimated from the
     background model.
   \item \Argument Value
   \item \Default Not set
   \elist

\subsubsection{\hbracket{vp-vs-ratio-from-wells}}\newkw{vp-vs-ratio-from-wells}
 \slist
   \item \Description If this command is given, the Vp/Vs ratio used
     in the reflection matrix will be estimated from well data. By
     default, the ratio is taken from the background model. If the
     keyword \kw{wavelet-estimation-interval} has also been specified,
     the estimate will be limited to that interval.
   \item \Argument 'yes' or 'no'
   \item \Default 'no'
   \elist

\subsubsection{\hbracket{maximum-relative-thickness-difference}}\newkw{maximum-relative-thickness-difference}
 \slist
   \item \Description Value giving the limit of how small the minimum interval thickness can be relative to maximum. If this gets too low, the transformation to stationarity for the FFT-algorithm gives strange results.
   \item \Argument Value
   \item \Default Default is 0.5, which is acceptable. Slightly smaller seems to work as well.
 \elist

\subsubsection{\hbracket{frequency-band}}\newkw{frequency-band}
 \slist
   \item \Description This command controls the frequency band of the inversion, so high and/or low frequencies can be filtered away. This ought to be done by the wavelet, but can be done here.
   \item \Argument Elements controlling the frequency band of the inversion
   \item \Default
 \elist

\paragraph{\hbracket{low-cut}}\newkw{low-cut}
 \slist
   \item \Description Value setting the minimum frequency affected by the inversion.
   \item \Argument Value
   \item \Default 5.0
 \elist

\paragraph{\hbracket{high-cut}}\newkw{high-cut}
 \slist
   \item \Description Value setting the maximum frequency affected by the inversion.
   \item \Argument Value
   \item \Default 55.0
 \elist

\subsubsection{\hbracket{energy-threshold}}\newkw{energy-threshold}
 \slist
   \item \Description If the energy in a trace falls below this threshold relative to the average, the trace is interpolated from neighbours.
   \item \Argument Value
   \item \Default 0.0
 \elist

\subsubsection{\hbracket{wavelet-tapering-length}}\newkw{wavelet-tapering-length}
 \slist
   \item \Description Value giving the length of the wavelet to be estimated in ms. For a 1D wavelet this is the width of the Papoulis taper used, when 
tapering of auto-correlation of reflection coefficients, and cross-correlation of reflection-coefficients and seismic data,
before doing the spectral division.  Increasing this value will give less bias (i.e. less energy at low frequencies), but larger 
variance in the estimate (i.e. secondary oscillations).   For a 3D wavelet this gives the length of the source-wavelet, by dividing this length by 
the vertical sampling interval we find the the number of parameters that is estimated  by the least-squares approach in the 3D-wavelet. 
   \item \Argument Value
   \item \Default 200.0
 \elist

\subsubsection{\hbracket{minimum-relative-wavelet-amplitude}}\newkw{minimum-relative-wavelet-amplitude}
 \slist
   \item \Description Value giving the ratio between the smallest relevant amplitude and the largest amplitude of peaks on an estimated wavelet. Edge peaks below this ratio are removed.
   \item \Argument Value
   \item \Default 0.05
 \elist

\subsubsection{\hbracket{maximum-wavelet-shift}}\newkw{maximum-wavelet-shift}
 \slist
   \item \Description Value controlling how much the wavelet is allowed to be shifted when doing estimation of wavelet or noise.
   \item \Argument Value
   \item \Default 11.0
 \elist

\subsubsection{\hbracket{minimum-sampling-density}}\newkw{minimum-sampling-density}
 \slist
   \item \Description Threshold value for minimum sampling density allowed.
   \item \Argument Value
   \item \Default 0.5 ms
 \elist

\subsubsection{\hbracket{minimum-horizontal-resolution}}\newkw{minimum-horizontal-resolution}
 \slist
   \item \Description Threshold value for minimum horizontal resolution allowed.
   \item \Argument Value
   \item \Default 5 m
 \elist

 \subsubsection{\hbracket{white-noise-component}}\newkw{white-noise-component}
 \slist
   \item \Description In order to stabilise the inversion, we need to interpret some of the noise as white. This value controls the fraction.
   \item \Argument Value between 0 and 1.
   \item \Default 0.1
 \elist

\subsubsection{\hbracket{reflection-matrix}}\newkw{reflection-matrix}
\slist
   \item \Description The file should be a 3 by  number of seismic data cubes ascii matrix. The first column is the factor used for vp for each cube setting (angle and ps/pp) when computing the reflection coefficients. The second and third are for vs and density.
   \item \Argument File name
   \item \Default Linearised Aki-Richards.
 \elist

\subsubsection{\hbracket{kriging-data-limit}}\newkw{kriging-data-limit}
 \slist
   \item \Description Integer value giving the limit for the amount of well data used to krige each point. A high value gives a smoother and more exact field, but takes more time.
   \item \Argument Integer
   \item \Default 250
 \elist

\subsubsection{\hbracket{debug-level}}\newkw{debug-level}
 \slist
   \item \Description Gives debug messages and output. Not intended for use except on request by NR.
   \item \Argument Integer value 0, 1 or 2.
   \item \Default 0
 \elist

\subsubsection{\hbracket{smooth-kriged-parameters}}\newkw{smooth-kriged-parameters}
\slist
   \item \Description Tells whether we should smooth borders between kriging blocks or not.
   \item \Argument yes or no
   \item \Default no
\elist

\subsubsection{\hbracket{rms-panel-mode}}\newkw{rms-panel-mode}
\slist
   \item \Description Disables some checks that are unnecessary when
   running from RMS.
   \item \Argument yes or no
   \item \Default no
\elist

\subsubsection{\hbracket{guard-zone}}\newkw{guard-zone}
\slist
   \item \Description Changes the amount of data that will be required
     outside the interval of interest. The guard zone contains data
     that will contribute to the inversion results in the interval of
     interest, and reducing the guard zone is therefore discouraged.
   \item \Argument Value
   \item \Default 100ms
\elist

\subsubsection{\hbracket{3d-wavelet-tuning-factor}}\newkw{3d-wavelet-tuning-factor}
\slist
   \item \Description Allows tuning of the 3D wavelet estimation. A
   large value forces better fit of wavelet, while a smaller value
   gives smaller secondary peaks in wavelet.
   \item \Argument Value
   \item \Default 50
\elist

\subsubsection{\hbracket{gradient-smoothing-range}}\newkw{gradient-smoothing-range}
\slist
   \item \Description Controls smoothing of the gradient used in 3D
     wavelet estimate and 3D inversion. This should be of the order of
     the horisontal extent of the 3D wavelet.
   \item \Argument Value
   \item \Default 100
\elist

\subsubsection{\hbracket{estimate-well-gradient-from-seismic}}\newkw{estimate-well-gradient-from-seismic}
\slist
   \item \Description If 'yes', the well gradient used for 3D wavelet
     estimation is estimated from seismic. Otherwise this is taken
     from the correlation direction.
   \item \Argument yes or no
   \item \Default no
\elist

%%%%%%%%%%%%%%%%%%%%%%%%%%%%%%%%%%%%%%%%%%%%%%%%%%%%%%%%%%%%%%%%%%%%%%%%%
%%%%%                             SURVEY                            %%%%%
%%%%%%%%%%%%%%%%%%%%%%%%%%%%%%%%%%%%%%%%%%%%%%%%%%%%%%%%%%%%%%%%%%%%%%%%%

 \section{\hbracket{survey}\necessary} \newkw{survey}
 \slist
   \item \Description All information about the seismic data is collected here.
   \item \Argument Elements containing information about the seismic data
   \item \Default
 \elist

 \subsection{\hbracket{angular-correlation}}\newkw{angular-correlation}
 \slist
   \item \Description 1D variogram. Gives the noise correlation between survey angles.
   \item \Argument 1D variogram, see \autoref{sec:variogram}.
   \item \Default
 \elist

 \subsection{\hbracket{segy-start-time}}\newkw{segy-start-time}
 \slist
   \item \Description Global start time for segy cubes. This is used if no individual time is given for a segy-cube.
   \item \Argument Value
   \item \Default
 \elist

\subsection{\hbracket{angle-gather}\necessary}\newkw{angle-gather}
 \slist
   \item \Description Repeatable command, one for each seismic data cube.
   \item \Argument Elements containing information about the different seismic data cubes
   \item \Default
 \elist

\subsubsection{\hbracket{offset-angle}\necessary}\newkw{offset-angle}
 \slist
   \item \Description This is the angle for the seismic data cube.
   \item \Argument Value
   \item \Default
 \elist

\subsubsection{\hbracket{seismic-data}\necessary}\rnewkw{seismic-data}{seismic-data2}
 \slist
   \item \Description Information about the seismic data cube.
   \item \Argument Elements containing information about the seismic data cube
   \item \Default
 \elist

\paragraph{\hbracket{file-name}\necessary}\newkw{file-name}
 \slist
   \item \Description File name for the seismic data cube of one of the standard 3D formats, see \autoref{sec:gridformats}. The file
     type will be automatically detected.
   \item \Argument File name
   \item \Default
 \elist

\paragraph{\hbracket{start-time}}\newkw{start-time}
 \slist
   \item \Description Value giving the start time for this segy file.
   \item \Argument Value
   \item \Default
 \elist

\paragraph{\hbracket{segy-format}}\rnewkw{segy-format}{segy-format2}
 \slist
   \item \Description Information about the segy format. By default, CRAVA recognises SeisWorks, IESX, SIP and Charisma. See Table \ref{tab:segyformats}.
   \item \Argument Elements containing information about the segy format
   \item \Default
 \elist

\subparagraph{\hbracket{standard-format}}\rnewkw{standard-format}{standard-format2}
 \slist
   \item \Description Giving the starting format for modifications.
   \item \Argument 'seisworks', 'iesx', 'charisma' or 'SIP'
   \item \Default 'seisworks'
 \elist

\subparagraph{\hbracket{location-x}}\rnewkw{location-x}{location-x2}
 \slist
   \item \Description The byte location for the x-coordinate in the trace header.
   \item \Argument Integer
   \item \Default
 \elist

\subparagraph{\hbracket{location-y}}\rnewkw{location-y}{location-y2}
 \slist
   \item \Description The byte location for the y-coordinate in the trace header.
   \item \Argument Integer
   \item \Default
 \elist

\subparagraph{\hbracket{location-il}}\rnewkw{location-il}{location-il2}
 \slist
   \item \Description The byte location for the inline in the trace header.
   \item \Argument Integer
   \item \Default
 \elist

\subparagraph{\hbracket{location-xl}}\rnewkw{location-xl}{location-xl2}
 \slist
   \item \Description The byte location for the crossline in the trace header.
   \item \Argument Integer
   \item \Default
 \elist

\subparagraph{\hbracket{bypass-coordinate-scaling}}\rnewkw{bypass-coordinate-scaling}{bypass-coordinate-scaling2}
 \slist
   \item \Description Indicates whether coordinate scaling information should be used.
   \item \Argument 'yes' or 'no'
   \item \Default
 \elist

\subparagraph{\hbracket{location-scaling-coefficient}}\rnewkw{location-scaling-coefficient}{location-scaling-coefficient2}
 \slist
   \item \Description
   \item \Argument Integer
   \item \Default
 \elist

\paragraph{\hbracket{type}}\newkw{type}
 \slist
   \item \Description Indicating the type of seismic data. Note that if both pp and ps cubes are used, these must be aligned.
   \item \Argument 'pp' or 'ps'
   \item \Default 'pp'
 \elist

\subsubsection{\hbracket{wavelet}}\newkw{wavelet}
 \slist
   \item \Description Information about the wavelet for this angle and seismic type. If not given, the wavelet will be estimated.
   \item \Argument Elements containing information about the wavelet and seismic type
   \item \Default
 \elist

\paragraph{\hbracket{file-name}}\rnewkw{file-name}{file-name2}
 \slist
   \item \Description File name for wavelet file on JASON or NORSAR format. Can not be given together with ricker. If neither file-name nor ricker is given, wavelet is estimated.
   \item \Argument File name
   \item \Default
 \elist

\paragraph{\hbracket{ricker}}\newkw{ricker}
 \slist
   \item \Description Use Ricker wavelet. Can not be given together with file-name. If neither file-name nor ricker is given, wavelet is estimated.
   \item \Argument Peak frequency
   \item \Default
 \elist

\paragraph{\hbracket{scale}}\newkw{scale}
 \slist
   \item \Description Wavelet read from file or ricker wavelet is multiplied by this. Has no meaning when wavelet is estimated.
   \item \Argument Value
   \item \Default
 \elist

\paragraph{\hbracket{estimate-scale}}\newkw{estimate-scale}
 \slist
   \item \Description Should global scale be estimated?
   \item \Argument 'yes' or 'no'
   \item \Default
 \elist

\paragraph{\hbracket{local-wavelet}}\newkw{local-wavelet}
 \slist
   \item \Description The amplitude and shift of the wavelet may be modified locally by 2D fields for shift and scale values. This is handled here.
   \item \Argument Elements modifying the amplitude and shift of the wavelet
   \item \Default
 \elist

\subparagraph{\hbracket{shift-file}}\newkw{shift-file}
 \slist
   \item \Description File name for standard surface file giving the local shift for the wavelet. Not allowed when wavelet is estimated.
   \item \Argument File name
   \item \Default
 \elist

\subparagraph{\hbracket{scale-file}}\newkw{scale-file}
 \slist
   \item \Description File name for standard surface file giving local scale for the wavelet. Not allowed when wavelet is estimated.
   \item \Argument File name
   \item \Default
 \elist

\subparagraph{\hbracket{estimate-shift}}\newkw{estimate-shift}
 \slist
   \item \Description Should a local shift be estimated? Not allowed with \kw{shift-file}, but can be used both with given and estimated wavelet.
   \item \Argument 'yes' or 'no'
   \item \Default
 \elist

\subparagraph{\hbracket{estimate-scale}}\rnewkw{estimate-scale}{estimate-scale2}
 \slist
   \item \Description Should a local scale be estimated? Not allowed with \kw{scale-file}, but can be used both with given and estimated wavelet.
   \item \Argument 'yes' or 'no'
   \item \Default
 \elist

\subsubsection{\hbracket{wavelet-3d}}\newkw{wavelet-3d}
 \slist
   \item \Description Information about the 3D-wavelet for this angle and seismic type. If not given, a 1D-wavelet is assumed.
   \item \Argument Elements containing information about the 3D-wavelet
   \item \Default
 \elist

\paragraph{\hbracket{file-name}}\rnewkw{file-name}{file-name4}
 \slist
   \item \Description File name for the 1D-wavelet file. This 1D-wavelet will define the 3D-wavelet in combination with the filter given in \kw{processing-factor-file-name}. If not given, the 1D-wavelet is estimated.
   \item \Argument File name
   \item \Default
 \elist

\paragraph{\hbracket{processing-factor-file-name}}\newkw{processing-factor-file-name}
 \slist
   \item \Description File name for 3D-wavelet damping factor filter file. The 3D-wavelet is defined by the 1D-wavelet and the filter. The 1D-wavelet is either given in \rkw{file-name}{file-name4} or estimated. In either case this filter file must be given.
   \item \Argument File name for the amplitude scalings in the wavenumber filter.
   \item \Default
 \elist

\paragraph{\hbracket{propagation-factor-file-name}}\newkw{propagation-factor-file-name}
 \slist
   \item \Description File name for 3D-wavelet correction filter file. This is used to set up the noise model for the 3D-wavelet.
   \item \Argument File name for the correction factors in the wavenumber filter.
   \item \Default
 \elist

\paragraph{\hbracket{stretch-factor}}\newkw{stretch-factor}
 \slist
   \item \Description Stretch factor for 3D-wavelet. The pulse is stretch with this factor.
   \item \Argument Value > 0.0
   \item \Default 1.0
\elist

\paragraph{\hbracket{estimation-range-x-direction}}\newkw{estimation-range-x-direction}
 \slist
   \item \Description Range for area around the well in x-direction where data are used in 3D wavelet estimation.
   \item \Argument Value >= 0.0
   \item \Default 0.0
\elist

\paragraph{\hbracket{estimation-range-y-direction}}\newkw{estimation-range-y-direction}
 \slist
   \item \Description Range for area around the well in y-direction where data are used in 3D wavelet estimation.
   \item \Argument Value >= 0.0
   \item \Default 0.0
\elist

\subsubsection{\hbracket{match-energies}}\newkw{match-energies}
 \slist
   \item \Description If 'yes', signal to noise ratio and wavelet scaling will be set to match model values with empirical values. Not a common estimator.
   \item \Argument 'yes' or 'no'
   \item \Default
 \elist

\subsubsection{\hbracket{signal-to-noise-ratio}}\newkw{signal-to-noise-ratio}
 \slist
   \item \Description Value for the signal to noise value. If not given, this will be estimated.
   \item \Argument Value
   \item \Default
 \elist

\subsubsection{\hbracket{local-noise-scaled}}\newkw{local-noise-scaled}
 \slist
   \item \Description Name of standard surface file with local noise.
   \item \Argument File name
   \item \Default
 \elist

\subsubsection{\hbracket{estimate-local-noise}}\newkw{estimate-local-noise}
 \slist
   \item \Description Can not say 'yes' here if \kw{local-noise-scaled} is given.
   \item \Argument 'yes' or 'no'
   \item \Default
 \elist

\subsection{\hbracket{wavelet-estimation-interval}}\newkw{wavelet-estimation-interval}
 \slist
   \item \Description Controls the time interval used for wavelet estimation by a top and base surface.
   \item \Argument Elements controlling the time interval used for wavelet estimation
   \item \Default By default, estimation is done from all available seismic and well data.
 \elist

\subsubsection{\hbracket{top-surface-file}}\newkw{top-surface-file}
 \slist
   \item \Description File name for standard surface file giving the top of the time interval used for wavelet estimation.
   \item \Argument File name
   \item \Default
 \elist

\subsubsection{\hbracket{base-surface-file}}\newkw{base-surface-file}
 \slist
   \item \Description File name for standard surface file giving the base of the time interval used for wavelet estimation.
   \item \Argument File name
   \item \Default
 \elist

\subsection{\hbracket{time-gradient-settings}}\newkw{time-gradient-settings}
 \slist
   \item \Description The prior standard deviation of the gradients are given, as well as the minimum distance for where gradient lines should not cross each other
   \item \Argument The standard deviation and the minimum distance
   \item \Default
 \elist

\subsubsection{\hbracket{distance}}\newkw{distance}
 \slist
   \item \Description The minimum lateral distance for where the gradient lines should not cross. The distance is equal for both x- and y-direction.
   \item \Argument Value
   \item \Default 100 m
 \elist

\subsubsection{\hbracket{sigma}}\newkw{sigma}
 \slist
   \item \Description The prior standard deviation of the gradient. Equal for both x- and y-gradients
   \item \Argument Value
   \item \Default 1 ms/m
 \elist

%%%%%%%%%%%%%%%%%%%%%%%%%%%%%%%%%%%%%%%%%%%%%%%%%%%%%%%%%%%%%%%%%%%%%%%%%
%%%%%                           WELL DATA                           %%%%%
%%%%%%%%%%%%%%%%%%%%%%%%%%%%%%%%%%%%%%%%%%%%%%%%%%%%%%%%%%%%%%%%%%%%%%%%%

\section{\hbracket{well-data}} \newkw{well-data}
 \slist
   \item \Description All information about the well data is collected here.
   \item \Argument Elements containing information about the well data
   \item \Default
 \elist

\subsection{\hbracket{log-names}} \newkw{log-names}
 \slist
   \item \Description CRAVA needs to find the time, vp, vs, density and possibly facies logs. The name of these logs in the well files are given here.
   \item \Argument Name of logs
   \item \Default
 \elist

\subsubsection{\hbracket{time}} \rnewkw{time}{time2}
 \slist
   \item \Description Name of the TWT log
   \item \Argument String
   \item \Default %'TWT'
 \elist

\subsubsection{\hbracket{vp}}\rnewkw{vp}{vp2}
 \slist
   \item \Description Name of the vp log. May not be given if \kw{dt} is given.
   \item \Argument String
   \item \Default
 \elist

\subsubsection{\hbracket{dt}}\newkw{dt}
 \slist
   \item \Description Name of the inverse vp log. May not be given if \rkw{vp}{vp2} is given.
   \item \Argument String
   \item \Default %'DT'
 \elist

\subsubsection{\hbracket{vs}}\rnewkw{vs}{vs2}
 \slist
   \item \Description Name of the vs log. May not be given if \kw{dts} is given.
   \item \Argument String
   \item \Default
 \elist

\subsubsection{\hbracket{dts}}\newkw{dts}
 \slist
   \item \Description Name of the inverse vs log. May not be given if \rkw{vs}{vs2} is given.
   \item \Argument String
   \item \Default %'DTS'
 \elist

\subsubsection{\hbracket{density}}\rnewkw{density}{density2}
 \slist
   \item \Description Name of the density log.
   \item \Argument String
   \item \Default %'RHOB'
 \elist

\subsubsection{\hbracket{facies}}\newkw{facies}
 \slist
   \item \Description Name of the facies log.
   \item \Argument String
   \item \Default %'FACIES'
 \elist

\subsection{\hbracket{well}}\newkw{well}
 \slist
   \item \Description Repeatable command, one for each well. Contains information about the wells.
   \item \Argument Elements containing information about the wells
   \item \Default
 \elist

\subsubsection{\hbracket{file-name}}\rnewkw{file-name}{file-name3}
 \slist
   \item \Description File name for a well file. RMS or Norsar format.
   \item \Argument File name
   \item \Default
 \elist


\subsubsection{\hbracket{use-for-wavelet-estimation}}\newkw{use-for-wavelet-estimation}
 \slist
   \item \Description Should this well be used for wavelet estimation?
   \item \Argument 'yes' or 'no'
   \item \Default
 \elist

\subsubsection{\hbracket{use-for-background-trend}}\newkw{use-for-background-trend}
 \slist
   \item \Description Should this well be used for background trend estimation?
   \item \Argument 'yes' or 'no'
   \item \Default
 \elist

\subsubsection{\hbracket{use-for-facies-probabilities}}\newkw{use-for-facies-probabilities}
 \slist
   \item \Description Should this well be used for facies probability estimation?
   \item \Argument 'yes' or 'no'
   \item \Default
 \elist

\subsubsection{\hbracket{synthetic-vs-log}}\newkw{synthetic-vs-log}
 \slist
   \item \Description Is the Vs log in this well synthetic? Will be detected from Vp correlation if not specified here.
   \item \Argument 'yes' or 'no'
   \item \Default
 \elist

\subsubsection{\hbracket{filter-elastic-logs}}\newkw{filter-elastic-logs}
 \slist
   \item \Description Should we multi-parameter-filter the elastic
                      logs in this well after the inversion?
   \item \Argument 'yes' or 'no'
   \item \Default
 \elist

\subsubsection{\hbracket{optimize-position}}\newkw{optimize-position}
 \slist
   \item \Description Repeatable command, one for each offset angle used for estimating optimised well location for this well.
   \item \Argument Elements controlling optimisation of well location
   \item \Default
 \elist

\paragraph{\hbracket{angle}}\rnewkw{angle}{angle3}
 \slist
    \item \Description Offset angle used for estimating optimised well location
    \item \Argument Value
    \item \Default
 \elist

\paragraph{\hbracket{weight}}\newkw{weight}
 \slist
    \item \Description Weight of the offset angle given in \rrkw{angle}{angle2}{angle3}
    \item \Argument Value
    \item \Default 1
 \elist

\subsection{\hbracket{high-cut-seismic-resolution}}\newkw{high-cut-seismic-resolution}
 \slist
   \item \Description This frequency is used to filter wells down to seismic resolution. Only used to generate output logs for QC.
   \item \Argument Value
   \item \Default
 \elist

\subsection{\hbracket{allowed-parameter-values}}\newkw{allowed-parameter-values}
 \slist
   \item \Description Sometimes there are faulty values in well logs. Here, trigger values for error detection can be controlled. These fall in two categories: Actual log values that are wrong, or logs that have extremely low or high variance when the background model is subtracted.
   \item \Argument Elements controlling trigger values for error detection
   \item \Default
 \elist

\subsubsection{\hbracket{minimum-vp}}\newkw{minimum-vp}
 \slist
   \item \Description Value for the smallest legal vp value.
   \item \Argument Value
   \item \Default 1300 m/s
 \elist

\subsubsection{\hbracket{maximum-vp}}\newkw{maximum-vp}
 \slist
   \item \Description Value for the largest legal vp value.
   \item \Argument Value
   \item \Default 7000 m/s
 \elist

\subsubsection{\hbracket{minimum-vs}}\newkw{minimum-vs}
 \slist
   \item \Description Value for the smallest legal vs value.
   \item \Argument Value
   \item \Default 200 m/s
 \elist

\subsubsection{\hbracket{maximum-vs}}\rnewkw{maximum-vs}{maximum-vs2}
 \slist
   \item \Description Value for the largest legal vs value.
   \item \Argument Value
   \item \Default 4200 m/s
 \elist

\subsubsection{\hbracket{minimum-density}}\newkw{minimum-density}
 \slist
   \item \Description Value for the smallest legal density value.
   \item \Argument Value
   \item \Default 1.4 g/cm$^3$
\elist

\subsubsection{\hbracket{maximum-density}}\newkw{maximum-density}
 \slist
   \item \Description Value for the largest legal density value.
   \item \Argument Value
   \item \Default 3.3 g/cm$^3$
 \elist

\subsubsection{\hbracket{minimum-variance-vp}}\newkw{minimum-variance-vp}
 \slist
   \item \Description Value for the smallest legal variance in the vp log after the background is subtracted and logarithm is taken.
   \item \Argument Value
   \item \Default 0.0005
 \elist

\subsubsection{\hbracket{maximum-variance-vp}}\newkw{maximum-variance-vp}
 \slist
   \item \Description Value for the largest legal variance in the vp log after the background is subtracted and logarithm is taken.
   \item \Argument Value
   \item \Default 0.0250
 \elist

\subsubsection{\hbracket{minimum-variance-vs}}\newkw{minimum-variance-vs}
 \slist
   \item \Description Value for the smallest legal variance in the vs log after the background is subtracted and logarithm is taken.
   \item \Argument Value
   \item \Default 0.0010
 \elist

\subsubsection{\hbracket{maximum-variance-vs}}\newkw{maximum-variance-vs}
 \slist
   \item \Description Value for the largest legal variance in the vs log after the background is subtracted and logarithm is taken.
   \item \Argument Value
   \item \Default 0.0500
 \elist

\subsubsection{\hbracket{minimum-variance-density}}\newkw{minimum-variance-density}
 \slist
   \item \Description Value for the smallest legal variance in the density log after the background is subtracted and logarithm is taken.
   \item \Argument Value
   \item \Default 0.0002
 \elist

\subsubsection{\hbracket{maximum-variance-density}}\newkw{maximum-variance-density}
 \slist
   \item \Description Value for the largest legal variance in the density log after the background is subtracted and logarithm is taken.
   \item \Argument Value
   \item \Default 0.0100
 \elist

\subsubsection{\hbracket{minimum-vp-vs-ratio}}\newkw{minimum-vp-vs-ratio}
 \slist
   \item \Description Value for the smallest Vp/Vs-ratio regarded as likely.
   \item \Argument Value
   \item \Default 1.4
 \elist

\subsubsection{\hbracket{maximum-vp-vs-ratio}}\newkw{maximum-vp-vs-ratio}
 \slist
   \item \Description Value for the largest Vp/Vs-ratio regarded as likely.
   \item \Argument Value
   \item \Default 3.0
 \elist

\subsection{\hbracket{maximum-deviation-angle}}\newkw{maximum-deviation-angle}
 \slist
   \item \Description Value for the maximum deviation angle of a well before it is excluded from estimation based on vertical wells (such as wavelet and signal to noise).
   \item \Argument Value
   \item \Default 15
 \elist

\subsection{\hbracket{maximum-rank-correlation}}\newkw{maximum-rank-correlation}
 \slist
   \item \Description If the correlation between vp and vs logs exceed this value, the vs log is considered to be synthetic, and not counted as additional data in estimation.
   \item \Argument Value close to, but less than 1.
   \item \Default 0.99
 \elist

\subsection{\hbracket{maximum-merge-distance}}\newkw{maximum-merge-distance}
 \slist
   \item \Description Value giving the minimum distance in time between well log entries before they are merged to one observation.
   \item \Argument Value
   \item \Default 0.01
 \elist

\subsection{\hbracket{maximum-offset}}\newkw{maximum-offset}
 \slist
   \item \Description Value giving the maximum allowed offset for moving wells in meters.
   \item \Argument Value
   \item \Default 250
 \elist

\subsection{\hbracket{maximum-shift}}\newkw{maximum-shift}
 \slist
   \item \Description Value giving the maximum allowed vertical shift for moving wells.
   \item \Argument Value
   \item \Default 11.0
 \elist

\subsection{\hbracket{well-move-data-interval}}\newkw{well-move-data-interval}
 \slist
   \item \Description Defines an interval for estimation of facies probability given elastic parameters.
   \item \Argument Elements defining estimation interval
   \item \Default Everywhere facies and elastic logs are present.
 \elist

\subsubsection{\hbracket{top-surface-file}}\rnewkw{top-surface-file}{top-surface-file2}
 \slist
   \item \Description File name for standard surface file giving the top of the estimation interval.
   \item \Argument File name
   \item \Default
 \elist

\subsubsection{\hbracket{base-surface-file}}\rnewkw{base-surface-file}{base-surface-file2}
 \slist
   \item \Description File name for standard surface file giving the base of the estimation interval.
   \item \Argument File name
   \item \Default
 \elist

%%%%%%%%%%%%%%%%%%%%%%%%%%%%%%%%%%%%%%%%%%%%%%%%%%%%%%%%%%%%%%%%%%%%%%%%%
%%%%%                         PRIOR MODEL                           %%%%%
%%%%%%%%%%%%%%%%%%%%%%%%%%%%%%%%%%%%%%%%%%%%%%%%%%%%%%%%%%%%%%%%%%%%%%%%%

\section{\hbracket{prior-model}}\newkw{prior-model}
 \slist
   \item \Description This command defines the prior model for elastic parameters and possibly also facies.
   \item \Argument Elements defining prior models for elastic parameters and facies
   \item \Default
 \elist

\subsection{\hbracket{background}}\rnewkw{background}{background3}
 \slist
   \item \Description Contains information about the background model or how to estimate it. Note that either all parameters must be given, or all must be estimated.
   \item \Argument Elements containing information about the background model
   \item \Default
 \elist

\subsubsection{\hbracket{ai-file}}\newkw{ai-file}
 \slist
   \item \Description File name for 3D grid file, giving background
   AI. Can not be given together with \kw{vp-file} or \kw{vp-constant}.
   \item \Argument File name
   \item \Default
 \elist

\subsubsection{\hbracket{si-file}}\newkw{si-file}
 \slist
   \item \Description File name for 3D grid file, giving background
   SI. Can not be given together with \kw{vs-file} or \kw{vs-constant}
   or \kw{vp-vs-ratio-file}.
   \item \Argument File name
   \item \Default
 \elist

\subsubsection{\hbracket{vp-vs-ratio-file}}\newkw{vp-vs-ratio-file}
 \slist
   \item \Description File name for 3D grid file, giving background
   Vp/Vs. Can not be given together with \kw{vs-file} or \kw{vs-constant}.
   \item \Argument File name
   \item \Default
 \elist

\subsubsection{\hbracket{vp-file}}\newkw{vp-file}
 \slist
   \item \Description File name for 3D grid file, giving background vp. Can not be given together with \kw{vp-constant}.
   \item \Argument File name
   \item \Default
 \elist

\subsubsection{\hbracket{vs-file}}\newkw{vs-file}
 \slist
   \item \Description File name for 3D grid file, giving background vs. Can not be given together with \kw{vs-constant}.
   \item \Argument File name
   \item \Default
 \elist

\subsubsection{\hbracket{density-file}}\newkw{density-file}
 \slist
   \item \Description File name for 3D grid file, giving background density. Can not be given together with \kw{density-constant}.
   \item \Argument File name
   \item \Default
 \elist

\subsubsection{\hbracket{vp-constant}}\newkw{vp-constant}
 \slist
   \item \Description Value, used for constant vp background. Can not be given together with \kw{vp-file}.
   \item \Argument Value
   \item \Default
 \elist

\subsubsection{\hbracket{vs-constant}}\newkw{vs-constant}
 \slist
   \item \Description Value, used for constant vs background. Can not be given together with \kw{vs-file}.
   \item \Argument Value
   \item \Default
 \elist

\subsubsection{\hbracket{density-constant}}\newkw{density-constant}
 \slist
   \item \Description Value, used for constant density background. Can not be given together with \kw{density-file}.
   \item \Argument Value
   \item \Default
 \elist

\subsubsection{\hbracket{velocity-field}}\rnewkw{velocity-field}{velocity-field2}
 \slist
   \item \Description File name for 3D grid file giving a velocity field used as base for vp in background estimation. Can not be used if the background parameters are given.
   \item \Argument File name
   \item \Default
 \elist

\subsubsection{\hbracket{lateral-correlation}}\newkw{lateral-correlation}
 \slist
   \item \Description 2D variogram for the lateral correlation in the elastic parameters in the estimated background model, used for kriging of wells. Can not be used if the background parameters are given.
   \item \Argument 2D variogram, see \autoref{sec:variogram}.
   \item \Default
 \elist

\subsubsection{\hbracket{high-cut-background-modelling}}\newkw{high-cut-background-modelling}
 \slist
   \item \Description Value giving the maximum frequency in the estimated background model. Can not be used if the background parameters are given.
   \item \Argument Value
   \item \Default 6.0 Hz
 \elist

\subsubsection{\hbracket{multizone-model}}\newkw{multizone-model}
 \slist
   \item \Description  Information about the multizone background model
   \item \Argument Elements containing information about the multizone background model
   \item \Default
 \elist

\paragraph{\hbracket{top-surface-file}}\rnewkw{top-surface-file}{top-surface-file4}
 \slist
   \item \Description Name of the top surface file used in multizone background estimation
   \item \Argument File name
   \item \Default
 \elist

\paragraph{\hbracket{top-surface-erosion-priority}}\newkw{top-surface-erosion-priority}
 \slist
   \item \Description Priority of the top surface file. The top surface file and all base surface files in the zones need to given different priority
   \item \Argument Int
   \item \Default
 \elist

\paragraph{\hbracket{zone}}\newkw{zone}
 \slist
   \item \Description Information about the different zones in the multizone background model
   \item \Argument Elements controlling the zones of the multizone background model
   \item \Default
 \elist

\subparagraph{\hbracket{base-surface-file}}\rnewkw{base-surface-file}{base-surface-file4}
 \slist
   \item \Description Name of a base surface file. The base surface files need to be given in descending order such that surfaces that are higher in the reservoir comes in before surfaces that are further down.
   \item \Argument File name
   \item \Default
 \elist

\subparagraph{\hbracket{erosion-priority}}\newkw{erosion-priority}
 \slist
   \item \Description Priority of the base surface file. The top surface file and all base surface files in the zones need to given different priority
   \item \Argument Int
   \item \Default
 \elist

\subparagraph{\hbracket{correlation-structure}}\newkw{correlation-structure}
 \slist
   \item \Description Correlation structure of the zone
   \item \Argument 'top', 'base' or 'compaction'
   \item \Default
 \elist

\subsection{\hbracket{earth-model}}\newkw{earth-model}
 \slist
   \item \Description Contains inverted seismic data used for forward modelling.
   \item \Argument Vp, vs and rho used for forward modelling.
   \item \Default
 \elist

\subsubsection{\hbracket{vp-file}}\rnewkw{vp-file}{vp-file2}
 \slist
   \item \Description File name for 3D grid file, giving vp.
   \item \Argument File name
   \item \Default
 \elist

\subsubsection{\hbracket{vs-file}}\rnewkw{vs-file}{vs-file2}
 \slist
   \item \Description File name for 3D grid file, giving vs.
   \item \Argument File name
   \item \Default
 \elist

\subsubsection{\hbracket{density-file}}\rnewkw{density-file}{density-file2}
 \slist
   \item \Description File name 3D grid file, giving density.
   \item \Argument File name
   \item \Default
 \elist

\subsubsection{\hbracket{ai-file}}\rnewkw{ai-file}{ai-file2}
 \slist
   \item \Description File name for 3D grid file, giving AI. Can not
     be given together with \kw{vp-file} .
   \item \Argument File name
   \item \Default
 \elist

\subsubsection{\hbracket{si-file}}\rnewkw{si-file}{si-file2}
 \slist
   \item \Description File name for 3D grid file, giving SI. Can not
   be given together with \kw{vs-file2} or \kw{vp-vs-ratio-file2}.
   \item \Argument File name
   \item \Default
 \elist

\subsubsection{\hbracket{vp-vs-ratio-file}}\rnewkw{vp-vs-ratio-file}{vp-vs-ratio-file2}
 \slist
   \item \Description File name for 3D grid file, giving Vp/Vs. Can
   not be given together with \kw{vs-file2}.
   \item \Argument File name
   \item \Default
 \elist

\subsection{\hbracket{local-wavelet}}\rnewkw{local-wavelet}{local-wavelet4}
 \slist
   \item \Description Contains prior information for local wavelet modelling.
   \item \Argument Elements containing prior information for local wavelet estimation.
   \item \Default
 \elist

\subsubsection{\hbracket{lateral-correlation}}\rnewkw{lateral-correlation}{lateral-correlation3}
 \slist
   \item \Description 2D variogram for the lateral correlation in local wavelet  modelling.
   \item \Argument 2D variogram, see \autoref{sec:variogram}.
   \item \Default
 \elist

\subsection{\hbracket{lateral-correlation}}\rnewkw{lateral-correlation}{lateral-correlation2}
 \slist
   \item \Description 2D variogram for the lateral correlation in the elastic parameters.
   \item \Argument 2D variogram, see \autoref{sec:variogram}.
   \item \Default
 \elist

\subsection{\hbracket{temporal-correlation}}\newkw{temporal-correlation}
 \slist
   \item \Description File name for the temporal correlation file. The file is an ascii file. Usually, this file comes from an earlier run of \crava.
   \item \Argument File name
   \item \Default
 \elist

\subsection{\hbracket{parameter-correlation}}\newkw{parameter-correlation}
 \slist
   \item \Description File name for the parameter correlation file. The file is an ascii file, containing covariances between Vp, Vs and density.
   It is most common to use a file resulting from an earlier run of \crava, and the file might look like this:

  \begin{verbatim}
    0.001171   0.000786   0.000046
  	0.000786   0.001810  -0.000395
  	0.000046  -0.000395   0.000530
  \end{verbatim}
   \item \Argument File name
   \item \Default
 \elist

\subsection{\hbracket{correlation-direction}}\newkw{correlation-direction}
 \slist
   \item \Description File name for the standard surface file giving the correlation direction for the inversion.
   \item \Argument File name
   \item \Default
 \elist

\subsection{\hbracket{facies-probabilities}}\rnewkw{facies-probabilities}{facies-probabilities2}
 \slist
   \item \Description Commands controlling the generating of facies probabilities.
   \item \Argument Elements controlling facies probabilities
   \item \Default
 \elist

\subsubsection{\hbracket{use-vs}}\newkw{use-vs}
\slist
  \item \Description Decides whether \vs information is used when computing facies probabilities.
  \item \Argument'yes' or 'no'.
  \item \Default 'yes'.
\elist

\subsubsection{\hbracket{use-prediction}}\newkw{use-prediction}
\slist
  \item \Description Decides whether sampled inversion logs are used when computing facies probabilities. If not, filtered logs are used.
  \item \Argument'yes' or 'no'.
  \item \Default 'no'.
\elist

\subsubsection{\hbracket{use-absolute-elastic-parameters}}  \newkw{use-absolute-elastic-parameters}
 \slist
   \item \Description Decides whether facies probabilities are
     generated based on absolute elastic parameters or elastic
     parameters minus trend (background model).
   \item \Argument 'yes' or 'no'
   \item \Default 'no'
 \elist

\subsubsection{\hbracket{estimation-interval}}\newkw{estimation-interval}
 \slist
   \item \Description Defines an interval for estimation of facies probability given elastic parameters.
   \item \Argument Elements defining estimation interval
   \item \Default Everywhere facies and elastic logs are present.
 \elist

\paragraph{\hbracket{top-surface-file}}{\rnewkw{top-surface-file}{top-surface-file3}
 \slist
   \item \Description File name for standard surface file giving the top of the estimation interval.
   \item \Argument File name
   \item \Default
 \elist

\paragraph{\hbracket{base-surface-file}}\rnewkw{base-surface-file}{base-surface-file3}
 \slist
   \item \Description File name for standard surface file giving the base of the estimation interval.
   \item \Argument File name
   \item \Default
 \elist

\subsubsection{\hbracket{prior-probabilities}}\newkw{prior-probabilities}
 \slist
   \item \Description Prior facies probabilities are given for all
     facies. Priors can be given as constant numbers or 3D
     cubes. If this command is not given, prior distribution is
     estimated from wells.
   \item \Argument Elements controlling facies probabilities
   \item \Default
 \elist

\paragraph{\hbracket{facies}}\rnewkw{facies}{facies2}
 \slist
   \item \Description Repeatable command, one for each facies. All facies present in well logs must be given.
   \item \Argument Elements containing information about the facies
   \item \Default
 \elist

\subparagraph{\hbracket{name}}\newkw{name}
 \slist
   \item \Description Name of facies.
   \item \Argument Facies name
   \item \Default
 \elist

\subparagraph{\hbracket{probability}}\newkw{probability}
 \slist
   \item \Description Probability for the facies given above. Either this command or \kw{probability-cube} is given, same for all facies.
   \item \Argument Real numbers between 0 and 1. Numbers for all facies must sum to one.
   \item \Default
 \elist

\subparagraph{\hbracket{probability-cube}}\newkw{probability-cube}
 \slist
   \item \Description File name for 3D grid file containing prior facies
     probability for facies with name given above. Either this command
     or \kw{probability} is given, same for all facies.
   \item \Argument File name
   \item \Default
 \elist


\subsubsection{\hbracket{uncertainty-level}}\newkw{uncertainty-level}
 \slist
   \item \Description Value defining how large the undefined
     probability will be when facies probabilities are computed. This
     value is scaled and used as likelihood for undefined when facies
     probabilities are computed.
   \item \Argument Value
   \item \Default 0.01
 \elist

%%%%%%%%%%%%%%%%%%%%%%%%%%%%%%%%%%%%%%%%%%%%%%%%%%%%%%%%%%%%%%%%%%%%%%%%%
%%%%%                            VARIOGRAM                          %%%%%
%%%%%%%%%%%%%%%%%%%%%%%%%%%%%%%%%%%%%%%%%%%%%%%%%%%%%%%%%%%%%%%%%%%%%%%%%

\section{Variogram}\newkw{variogram-keyword}
\label{sec:variogram}
  The variograms are given on the following form:

\subsection{\hbracket{variogram-type}}\newkw{variogram-type}
 \slist
   \item \Description Either 'genexp' or 'spherical' for general exponential or spherical variogram.
   \item \Argument
   \item \Default
 \elist

\subsection{\hbracket{angle}}\rnewkw{angle}{angle2}
 \slist
   \item \Description Value for the azimuth direction. Only for 2D variograms.
   \item \Argument
   \item \Default
 \elist

\subsection{\hbracket{range}}\newkw{range}
 \slist
   \item \Description Value for the range in the azimuth direction.
   \item \Argument
   \item \Default
 \elist

\subsection{\hbracket{subrange}}\newkw{subrange}
 \slist
   \item \Description Value for the range normal to the azimuth direction. Only for 2D variograms.
   \item \Argument
   \item \Default
 \elist

\subsection{\hbracket{power}}\newkw{power}
 \slist
   \item \Description Value between 1 and 2 for the power of the general exponential variogram. Not allowed for spherical variogram.
   \item \Argument
   \item \Default
 \elist

All angles are given as mathematical angles in degrees.




